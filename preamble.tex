\documentclass{report}
\usepackage[utf8]{inputenc}

\newif\ifdraft % whether to show guides, annotations, etc
\draftfalse

%% LAYOUT

% paper size, margins, etc
\usepackage[
    a5paper,
    twoside,
    left=10mm, right=10mm, % horizontal margins
    bindingoffset=10mm, % additional margin on binding edge
    top=1cm, includehead, % margin to top of header
    headsep=5mm, % distance from bottom of header to top of text
    % \ifdraft showframe, \else\fi % draw margin outlines and things
]{geometry}



%% CITATIONS
\usepackage{natbib}
\bibliographystyle{abbrvnat}
\setcitestyle{authoryear}
\renewcommand{\bibname}{References}



%% CHAPTERS

% format chapters as `1 Intro' instead of `Chapter 1 \\ Intro' and remove space above
\usepackage{titlesec}
\titleformat{\chapter}{\normalfont\Large\bfseries}{\thechapter}{1em}{}
\titlespacing*{\chapter}{0pt}{-10pt}{20pt}
\renewcommand{\thechapter}{\Roman{chapter}}



%% HEADERS
\usepackage{fancyhdr}

% define normal header style with chapter and section names
\fancypagestyle{headers}{ 
    \fancyhead{}{}{} % clear header
    \fancyfoot{}{}{} % clear footer
    \fancyhead[RO,LE]{\thepage} % add page numbers on outer corners
    \fancyhead[RE]{\leftmark} % add chapter title
    \fancyhead[LO]{\rightmark} % add section title
}

% define header style to be applied on chapter title pages
\fancypagestyle{plain}{ 
  \fancyhead{}{}{}
  \fancyfoot{}{}{}
  \fancyhead[RO,LE]{\thepage}
}

% select header style for normal pages
\pagestyle{headers}

% no line under header
\renewcommand{\headrulewidth}{0pt}

% omit page numbering until first chapter
\pagenumbering{gobble}

% chapter and section titles in headers
\renewcommand{\chaptermark}[1]{\markboth{\textsc{#1}}{}}
\renewcommand{\sectionmark}[1]{\markright{\arabic{chapter}.\arabic{section}\ \S\ #1}}

% start at chapter zero
\setcounter{chapter}{-1}



%% TYPOGRAPHY

% Allow exact control of font sizes
\usepackage{anyfontsize}
\usepackage{hyperref} % for digital versions
\usepackage{kantlipsum}
\usepackage[framemethod=default]{mdframed}

\newenvironment{note}[1][Note]{ % Environment for outlined 'aside boxes'
\vspace{-1.6ex}
\begin{mdframed}[
    frametitle={\colorbox{white}{\normalfont\emph{#1}}},
    frametitleaboveskip=-1.6ex,
    frametitlealignment=\raggedright,
    font=\small,
    linewidth=.3pt,
    leftmargin=1cm,
    rightmargin=1cm,]
\vspace{-1.6ex}
}{\end{mdframed}}
    
\usepackage[inline]{enumitem} % horizontal lists with starred environment
% \usepackage{enumerate} % intuitive labelling

% Notes to editor(s)
\newcommand{\alert}[1]{\ifdraft\textcolor{red}{\textbf{\underline{#1}}}\else\fi}



%% MATHS
\usepackage{amssymb}
\usepackage{mathtools} % provides: \Aboxed
\usepackage{physics}
\usepackage{accents}
\usepackage{dsfont}

\usepackage{empheq}
\usepackage{environ}

% Environment for boxed equations, use as e.g., `\begin{eqbox}{align}...\end{eqnbox}`
\NewEnviron{eqbox}[1]{\begin{empheq}[box=\fbox]{#1}\BODY\end{empheq}}

\newenvironment{baligned}{\left\{\begin{aligned}}{\end{\aligned}\right.}

%% SYMBOL MACROS

% David's tensor 'twiddle'
\newcommand{\ts}[1]{\symbf{\underaccent{\sim}{#1}}}

% Vector under-bar
% \newcommand{\uvec}[1]{\underaccent{\rightharpoondown}{#1}}
\newcommand{\uvec}{\vec}

% Minkowski space
\newcommand{\Mink}{\mathcal M}

% Nicer transposition superscript symbol
\newcommand{\transpose}{^\intercal}

% Groups written in \mathrm
\renewcommand{\O}{\mathrm{O}}
\newcommand{\SO}{\mathrm{SO}}
\newcommand{\ISO}{\mathrm{ISO}}

% Levi--Civita symbol
\newcommand{\leviciv}{\epsilon}

% "Such that" separator for set comprehensions in \qty{...}
\newcommand{\suchthat}{\;\middle|\;}

% 'constant' text for math
\newcommand{\const}{\text{const.}}

% Replace equation number label with right-aligned little note
\newcommand{\eqnote}[1]{\tag*{\small\emph{#1}}}

% WARNING: `autonum` is buggy as hell. Package must be loaded as late as possible, and equation labels may only contain ASCII
\usepackage{autonum} % show equation numbers only if they are referenced



%% GRAPHICS
\usepackage{graphicx}
\usepackage{wrapfig}

\usepackage{tikz}
    \usetikzlibrary{shapes,arrows,calc,backgrounds}
    \usetikzlibrary{external} 
    
    \tikzset{sketch/.append style={
        % inner frame sep=0pt,
        % \ifdraft framed, \else\fi
    }}

\tikzexternalize[prefix=tikzcache/]

\usepackage{pgfplots}
    \usepgfplotslibrary{colormaps}

\pgfplotsset{
    compat=1.8, % required for correct 3d label placement
    Frame/.style={
        axis lines=center,
        ticks=none,
        width=4cm,height=4cm,
        xmin=-1,xmax=1,
        ymin=-1,ymax=1,
        zmin=-1,zmax=1,
        view={30}{20},
        xlabel style={anchor=west},
        ylabel style={anchor=south west},
        zlabel style={anchor=south},
        xlabel={$x$},ylabel={$y$},zlabel={$z$},
        anchor=center,
    },
    Barred Frame/.style={
        Frame,
        xlabel={$\bar x$},ylabel={$\bar y$},zlabel={$\bar z$},
    }
}
