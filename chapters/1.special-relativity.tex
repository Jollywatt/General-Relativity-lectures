\chapter[Review of Special Relativity]{Review of Special Relativity}

\begin{note}[Background]
See \cite[ch~1]{schutz2009first} and \cite[ch 5, 12, 13]{doughty2018lagrangian}.
\end{note}

\noindent
Assumptions of Special Relativity:
\begin{enumerate}
	\item The world is described by a 4-dimensional continuum, \textbf{spacetime}, or \textbf{Minkowski space} $\Mink^4$, which is the set of all \textbf{events} $x^\mu$,
	\begin{align}
		x^\mu \equiv \ts x = \qty(x^0, x^i) \equiv \qty(x^0, \uvec x) = \qty(ct, \uvec x)
	.\end{align}
	\begin{note}[Notation]
	Greek indices, $\mu,\nu$ run over \textit{spacetime} index values; $\qty{0,1,2,3}$. \\
	Latin indices (mid-alphabet), $i,j,k$ run over \textit{spatial} index values; $\qty{1,2,3}$.
	$\ts x$ is a 4-vector (twiddle); $\uvec x$ is a 3-vector (under bar).
	\end{note}
	
	\item There exist \textbf{inertial frames}; namely, frames in which the measured values of time $t$ and position $x^i$ of events result in \textit{linear} equations of motion for \textit{free} particles.
\end{enumerate}


\noindent
Postulates of Special Relativity:
\begin{enumerate}
	\item \textsc{Principle of Relativity:}
	The laws of physics are invariant under transformations $x^\nu \to x^{\bar\mu}(x^\nu)$ from one inertial frame to another (and such transformations form a group).
	
	\item \textsc{Constancy of the Speed of Light:} There exists an invariant upper bound on all velocities
	\begin{align}
		\qty|\dv{\uvec x}{t}| \le \qty|\dv{\uvec x}{t}|_\text{max} = c
		\qqtext{(speed of light)}
	\end{align}
	and this value is the same in all inertial frames.
\end{enumerate}
	
For photons, \exercise
\begin{align}
	c = \underset{\text{frame $K$}}{\qty|\dv{\uvec x}{t}|} = \underset{\text{frame $K'$}}{\qty|\dv{\uvec x'}{t}|}
	\label{eqn:photon-speed}
.\end{align}
Rewriting \eqref{eqn:photon-speed} we see that, for photons, \exercise
\begin{align}
	\underset{\text{frame $K$}}{(\dd x)^2 + (\dd y)^2 + (\dd z)^2 - c^2(\dd t)^2}
	= \underset{\text{frame $K'$}}{(\dd x')^2 + (\dd y')^2 + (\dd z')^2 - c^2(\dd t')^2} = 0
.\end{align}
This suggests the definition of the \textbf{spacetime interval} between any neighbouring events $x^\mu$ and $x^\mu + \dd x^\mu$ as
\begin{align}
	\dd s^2 &\equiv -c^2\dd t^2 + \dd x^2 + \dd y^2 + \dd z^2
\\  &= \eta_{\mu\nu}\dd x^\mu \dd x^\nu
	\eqnote{...pseudo-Riemannian structure}
\intertext{where $x^\mu = (x^0, x^i) = (ct, x, y, z)$ and where}
	\eta_{\mu\nu} &= \smqty(\dmat{-1,+1,+1,+1})
	\equiv \operatorname{diag}\qty(-1, +1, +1, +1)
	\equiv -1 \oplus \mathds{1}_3
.\end{align}
For photons, $\dd s^2 = 0$.
Using the two postulates, one may show that the interval $\dd s^2$ is invariant with respect to coordinates based in any inertial frame \cite[\S1.6]{schutz2009first}.
\begin{align}
	\dd s^2 = \eta_{\mu\nu}\dd x^\mu \dd x^\nu = \eta_{\bar\mu\bar\nu}\dd x^{\bar\mu} \dd x^{\bar\nu}
	\label{eqn:def-spacetime-interval}
.\end{align}

\begin{note}
\begin{itemize}%[label=\arabic*)]
	\item The symbol $\dd s^2$ for the interval is purely notational convention, since we may have $\dd s^2 < 0$ is some cases.

	\item The first postulate alone implies either S.R.\ or its $c \to \infty$ limit (a.k.a.\ Galilean relativity).

	\item Formally, $\dd x^\mu\dd x^\nu$ is shorthand for the \textit{tensor product} $\dd x^\mu \otimes \dd x^\nu$, and you will see that $\dd s^2 = \ts\eta$ is really the \emph{metric tensor}...
\end{itemize}
\end{note}


\section{Lorentz Transformations}

In an inertial frame $K$, the equations of motion of a free particle are linear;
\begin{align}
	\dv[2]{x^\mu}{\lambda} &= 0
&   &\iff
&   x^\mu &= x_0^\mu + u^\mu\lambda
,\end{align}
where $x_0^\mu, u^\mu$ are constant and $\lambda$ is a parameter.
Similarly, in any other inertial frame $\bar K$,
\begin{align}
	\dv[2]{x^{\bar\mu}}{\lambda} &= 0
&   &\iff
&   x^{\bar\mu} &= x_0^{\bar\mu} + u^{\bar\mu}\lambda
.\end{align}
Now,
\begin{align}
	\dv{x^{\bar\mu}}{\lambda} &= \pdv{x^{\bar\mu}}{x^\nu}\dv{x^\nu}{\lambda} = \pdv{x^{\bar\mu}}{x^\nu}u^\nu
	\eqnote{...chain rule}
\\\implies
	0 &= \dv[2]{x^{\bar\mu}}{\lambda} = \pdv{x^\alpha}\qty(\pdv{x^{\bar\mu}}{x^\nu}u^\nu)u^\alpha
	= \pdv[2]{x^{\bar\mu}}{x^\alpha}{x^\nu}u^\nu u^\alpha
	\eqnote{$\because \; u^\nu$ constant}
\\\implies 0 &= \pdv[2]{x^{\bar\mu}}{x^\alpha}{x^\nu}
	\eqnote{...since true $\forall u^\alpha$}
.\end{align}
So the required transformation between inertial frames is \textit{linear};
\begin{eqbox}{align}
	x^{\bar\mu} = L^{\bar\mu}{}_\nu x^\nu + a^{\bar\mu}
	\label{eqn:Poincare-coord-transform}
,\end{eqbox}
where $a^{\bar\mu}$ and $L^{\bar\mu}{}_\nu \equiv \pdv{x^{\bar\mu}}{x^nu}$ are constants.
Differentiate \eqref{eqn:Poincare-coord-transform} and substitute into \eqref{eqn:def-spacetime-interval} to give
\begin{math}
	\eta_{\mu\nu}\dd x^\mu\dd x^\nu
	= \eta_{\bar\mu\bar\nu}L^{\bar\mu}{}_\mu L^{\bar\nu}{}_\nu\dd x^\mu\dd x^\nu
.\end{math}
Since this is true $\forall \, \dd x^\mu$,
\begin{eqbox}{align}
	\eta_{\bar\mu\bar\nu}L^{\bar\mu}{}_\mu L^{\bar\nu}{}_\nu
	= \eta_{\mu\nu}
	\label{eqn:Lorentz-matrix-condition}
.\end{eqbox}
In matrix form, \eqref{eqn:Poincare-coord-transform} and \eqref{eqn:Lorentz-matrix-condition} are
\begin{eqbox}{gather}
	\ts{\bar x} = L\ts{x} + \ts{a} \tag{\ref{eqn:Poincare-coord-transform}a}
,\\  L\transpose \eta L = \eta \tag{\ref{eqn:Lorentz-matrix-condition}a}
.\end{eqbox}

Transformations $\ts x \mapsto \ts{\bar x}$ defined by \eqref{eqn:Poincare-coord-transform},~\eqref{eqn:Lorentz-matrix-condition} are the \textit{inhomogeneous} Lorentz transformations, or \textbf{Poincaré transformations}, and form the Poincaré group $\mathrm{IO}(1,3)$ (pronounced Inhomogeneous Orthogonal group).



\begin{wrapfigure}[8]{r}{5cm}
	\vspace*{-\baselineskip}
	\begin{tikzpicture}[sketch]

\begin{axis}[
    Frame,
    name=K,
]
\end{axis}

\begin{axis}[
    Barred Frame,
    name=KBAR,
    at={($(K) + (15:3cm)$)},
]
\end{axis}

\draw[->,dashed] (K.center) --node[above] {$\ts a$} (KBAR.center);

\node[anchor=north west] at (.3, -.4) {\small \em (homogeneity of spacetime)};

\end{tikzpicture}
\end{wrapfigure}

``Inhomogeneous'' refers to the inclusion of spacetime translations
\begin{math}
	x^\nu \mapsto x^{\bar\mu} = \delta^{\bar\mu}{}_\nu x^\nu + a^{\bar\mu}
,\end{math}
which form a subgroup $\mathrm{T}^4$ of the Poincaré group.
If we set $a^{\bar\mu} = 0$ in \eqref{eqn:Poincare-coord-transform}, we are left with \textit{homogeneous} transformations, called simply the \textbf{Lorentz transformations}.


In G.R., our task is to generalise these ideas to general coordinate frames for which $L^{\bar\mu}{}_\nu$ are not necessarily constant.


\subsection{Examples of Lorentz Transformations}

A transformation belonging to the (homogeneous) Lorentz group $\O(1,3)$ can be represented as a matrix acting on coordinates $x^\mu$ when they are \emph{viewed as vectors.}
\begin{align}
	x^\mu \cong \ts{x} = \smqty(x^0 \\ x^1 \\ x^2 \\ x^3)
	= \smqty(t \\ x \\ y \\ z)
\end{align}
% Note that spacetime events (i.e., \textit{points} in spacetime) should not generally be thought of as \textit{vectors}---except when the spacetime is flat and the coordinates are expressed relative to a nearby orthogonal frame, as we assume here.

\begin{note}
	Spacetime events (i.e., \textit{points} in spacetime) are not themselves vectors---neither addition nor scalar multiplication of events makes physical sense (i.e., spacetime itself is not a \textit{vector space}).
	However, in S.R.\ we may represent events by their associated displacement vector relative to a chosen orthogonal inertial frame.
\end{note}

The Lorentz group $\O(1,3)$ consists of (combinations of) the following:
\begin{itemize}
	\item \textbf{Rotations,} e.g., about the $z$-axis (in the $xy$-plane) by an angle $\theta$;
	\begin{align}
		L_R(\theta) = \mqty(
			1 & 0 & 0 & 0 \\
			0 & \cos\theta & -\sin\theta & 0 \\
			0 & \sin\theta & \cos\theta & 0 \\
			0 & 0 & 0 & 1 \\
		)
	.\end{align}
	
	\item \textbf{Boosts,} e.g., by a velocity $\uvec v$ in the $x$-direction;
	\begin{align}
		L_B(\alpha) = \mqty(
			\cosh\alpha & -\sinh\alpha & 0 & 0 \\
			-\sinh\alpha & \cosh\alpha & 0 & 0 \\
			0 & 0 & 1 & 0 \\
			0 & 0 & 0 & 1 \\
		) = \mqty(
			\cos i\alpha & i\sin i\alpha & 0 & 0 \\
			i\sin i\alpha & \cos i\alpha & 0 & 0 \\
			0 & 0 & 1 & 0 \\
			0 & 0 & 0 & 1 \\
		)
	,\end{align}
	where \begin{math}
		\alpha = \tanh^{-1}\frac{v}{c}
	\end{math}
	is the \textit{rapidity parameter}.
	In terms of the velocity $\beta \equiv \frac{v}{c}$, one has $\cosh\alpha = \frac1{\sqrt{1 - \beta^2}} \equiv \gamma$ and $\sinh\alpha = \frac{\beta}{\sqrt{1 - \beta^2}} \equiv \beta\gamma$.
\end{itemize}

\begin{note}
Boosts appear similar to rotations, but differ as a consequence of the indefiniteness of the metric.
Formally, an $x$-boost of rapidity $\alpha$ is equivalent to a rotation by an `imaginary angle' $\alpha' = i\alpha$ through the $\tau x$-plane, where $\tau = it$ is `imaginary time'... though this is not a good picture physically!
\end{note}
An important difference between rotations and boosts is that, where $0 \le \theta < 2\pi$ for rotations, we have $-\infty < \alpha < \infty$ for boosts, i.e., rotations form a \textit{compact}\footnote{A \textit{compact} set is one for which any infinite sequence of elements contains a convergent subsequence. E.g., $[0, 2\pi)$ is compact, but $\mathbb{R}$ is not (consider the sequence $\qty{1, 2, 3, ...} \subset \mathbb{R}$).} subgroup of the Lorentz (or Poincaré) group, whereas boosts are \textit{non-compact}---and in fact do not form a subgroup (because, in general, the composition of two boosts forms a combination of a rotation and a boost).

Both rotations and boosts depend on continuous parameters ($\theta$ or $\alpha$).
However, the Lorentz group $\O(1,3)$ also contains \emph{discrete} transformations...
\begin{itemize}

	\item \textbf{Parity inversion};
	\begin{math}
		P = \operatorname{diag}(1, -\mathds{1}_3) \equiv
		\smqty(1 & & & \\ & -1 & & \\ & & -1 & \\ & & & -1)
	.\end{math}

	{\centering
		\begin{tikzpicture}[sketch]


\begin{axis}[
    Frame,
    name=K,
    view={30}{20},
]
\end{axis}

\begin{axis}[
    Barred Frame,
    name=KBAR,
    at={($(K) + (0:3.5cm)$)},
    view={180+30}{20},
    xlabel style={anchor=east},
    ylabel style={anchor=north east},
    zlabel style={anchor=north},
    z post scale=-1
]
\end{axis}

\draw[->] (K) --node[above] {$P$} (KBAR);

\end{tikzpicture}
	\\}
	Notice that $P$ is \textit{not} equivalent to a rotation; it transforms a right-handed frame into a left and vice versa, since $\det P = -1$.

	\begin{note}
	In even spatial dimensions, the transformation $R = \operatorname{diag}(1, -\mathds{1}_{2n})$ is \textit{not} a parity transformation; it is a rotation by $\pi$ and $\det R = 1$.
	In these cases, inversions $x_i \mapsto -x_i$ of a \textit{single} spatial coordinate are parity transformations.
	\end{note}

	\item \textbf{Time reversal};
	\begin{math}
		T = \operatorname{diag}(-1, \mathds{1}_3)
	.\end{math}
\end{itemize}

The Lorentz matrix condition \eqref{eqn:Lorentz-matrix-condition} implies that $(\det L)^2 = 1 \iff \det L = \pm 1$ for any Lorentz transformation $L \in \O(1,3)$.
Those with $\det L = +1$ and those with $\det L = -1$ form two disconnected pieces of $\O(1,3)$, but only the first piece contains the identity transformation $\delta^{\bar\mu}{}_\nu$.

Inspecting the $\bar\mu\nu = \bar00$ component of \eqref{eqn:Lorentz-matrix-condition} gives
\begin{align}
	\qty(L^{\bar0}{}_0)^2 - \sum_{\bar k = 1}^3\qty(L^{\bar k}{}_0)^2 = 1
	\implies
	\qty(L^{\bar0}{}_0)^2 \ge 1
,\end{align}
which shows that there exists two disconnected classes of Lorentz transformation with $L^{\bar0}{}_0 \ge 1$ and $L^{\bar0}{}_0 \le -1$.
Those with $L^{\bar0}{}_0 \ge 1$ are called \textbf{orthochronous}.

\subsection{The Restricted Lorentz Group}

We define the subgroup of restricted Lorentz transformations by adding two conditions to the Lorentz matrix condition \eqref{eqn:Lorentz-matrix-condition}; that they be
\begin{enumerate*}[label=\arabic*)]
	\item orthochronous and
	\item have determinant unity.
\end{enumerate*}
\begin{align}
	\SO^+(1,3) \equiv \qty{\Lambda \;\middle|\; \Lambda\transpose\eta\Lambda = \eta,
	\Lambda^{\bar0}{}_0 \ge 1, \det\Lambda = 1}
	\label{eqn:def-restricted-Lorentz-group}
\end{align}
We have removed the discrete transformations involving $ P$ and $ T$, so that $\SO^+(1,3)$ is \emph{continuous} and \emph{connected}, unlike $\O(1,3)$.

\begin{note}[Notation]
	The $\mathrm{S}$ in $\SO^+(1,3)$ refers to the condition $\det\Lambda = 1$, and the $^+$ refers to the orthochronous condition.
	Sometimes $\SO^+(1,3)$ is simply written as $\SO(1,3)$.
\end{note}

Groups whose elements may be continuously parametrised are \textbf{Lie groups}.
The translation group $\mathrm{T}^4$ (parametrised continuously by $\Delta x^\mu$) and the rotation group $\SO(3)$ (parametrised continuously by three angles) are examples of \emph{connected} Lie groups.

Reintroducing translations to the restricted Lorentz group gives the restricted Poincaré group $\mathrm{ISO}^+(1,3)$---also a connected Lie group.
Unrestricted groups may be reconstructed by reintroducing the discrete transformations;
\begin{align}
	\O(1,3) = \qty{\Lambda, \Lambda P, \Lambda T, \Lambda PT \mid \Lambda \in \SO^+(1,3)}
.\end{align}






\section{The Scalar Product in Minkowski Spacetime}






\section{The Causal Structure of Minkowski Spacetime}

Because of the indefiniteness of the Minkowski metric (and hence of the Minkowski scalar product), 4-vectors can have positive, zero or negative norm.
A 4-vector $\ts V$ is called
\begin{math}
	\begin{baligned}
		&\text{timelike}
	\\	&\text{null}
	\\	&\text{spacelike}
	\end{baligned}
\end{math}
if 
\begin{math}
	\ts V \cdot \ts V = V_\mu V^\mu = \eta_{\mu\nu}V^\mu V^\nu
	\begin{baligned}
		&< 0
	\\	&= 0
	\\	&> 0
	\end{baligned}
.\end{math}
% Similarly, a spacetime interval $\dd s$ is
% \begin{math}
% 	\begin{cases}
% 		\text{timelike} & \text{if $\dd s^2 < 0$} \\
% 		\text{null} & \text{if $\dd s^2 = 0$} \\
% 		\text{spacelike} & \text{if $\dd s^2 < 0$} \\
% 	\end{cases}
% \end{math}

\begin{wrapfigure}[16]{l}{6cm}
	\begin{tikzpicture}[
	point/.style={
		circle,
		fill,
		inner sep=1pt,
	},
	smol/.style={
		font=\small,
	}
]
	\begin{axis}[
		Frame,
		width=6cm,height=7cm,
		zlabel={$ct$},
		xlabel={$x^i$},
		ylabel={},
		z buffer = sort,
	]
	\addplot3[
		% surf,
		% shader = interp,
		smooth,
		opacity = 0.3,
		mesh,
		samples = 7,
		samples y = 17,
		domain = -1:1,
		domain y = 0:360,
		colormap={wbw}{rgb255=(255,255,255) rgb255=(0,0,0) rgb255=(255,255,255)},
	] ({x*sin(y)}, {x*cos(y)}, {x});

	\node[point] (O) at (axis cs:0,0,0) {};
	\node[point] (F) at (axis cs:0.3,0.3,1) {};
	\node[point] (E) at (axis cs:-0.3,-0.3,0) {};
	\node[point] (P) at (axis cs:-0.3,-0.3,-1) {};
	\end{axis}

	\node[smol,circle, inner sep=1pt, fill=white] at (O) {$P$};
	\node[smol,above right] at (F) {future of $P$};
	\node[smol,left] at (E) {elsewhere of $P$};
	\node[smol,below] at (P) {past of $P$};
\end{tikzpicture}
	\caption*{One spatial dimension suppressed; lightcone in $(3+1)$-d spacetime is a continuum of spheres.}
\end{wrapfigure}
By the second postulate, all \emph{causally connected} events relative to an event $P$ lie in its future or past lightcone, because $P$ cannot causally influence events outside the lightcone without transmitting superluminal data.

A particle's \textbf{worldline} is a curve $x^\mu = x^\mu(\lambda)$ whose tangent $\ts V$ given by $V^\mu = \dv{x^\mu}{\lambda}$ is everywhere timelike; $V^\mu V_\mu < 0$.
This implies that, in an inertial frame, the particles velocity $\uvec v$ is subluminal $|\uvec v| < c$ \exercise.

Consider a freely moving particle whose (timelike) worldline unit tangent vector is $\ts V$.
There always exists a Lorentz transformation which sends a timelike unit vector to $V^\mu \mapsto V^{\bar\mu} = (1, 0, 0, 0)$ \exercise.
The \textbf{proper time interval} $\dd\tau$ between two neighbouring events along the particle's worldline is defined as the interval of time measured in the particle's instantaneous rest frame.
\begin{align}
	c^2\dd\tau^2 = -\eta_{\mu\nu}\dd x^\mu\dd x^\nu
	\label{eqn:def-proper-time}
\end{align}

\begin{note}
	Since $\dd x$ is timelike, $\eta_{\mu\nu}\dd x^\mu\dd x^\nu < 0$, hence the minus sign.
	Proper time is not defined for spacelike intervals, since the above definition would yield an imaginary time.
	Instead, proper distance is defined by $\dd\ell^2 = \eta_{\mu\nu}\dd x^\mu\dd x^\nu$.
\end{note}
Proper time is the time \emph{experienced} by the particle; i.e., the time that would be measured by a clock moving on the same worldline.

Suppose we have a worldline $x^\mu(\lambda)$.
Integrate $\dd\tau$ via \eqref{eqn:def-proper-time} along the path to get the total proper time elapsed between events $x^\mu(\lambda_i)$ and $x^\mu(\lambda_f)$.
\begin{align}
	\Delta\tau = \frac1c \int_{\lambda_i}^{\lambda_f} \dd\lambda \, \sqrt{-\eta_{\mu\nu}\dv{x^\mu}{\lambda}\dv{x^\nu}{\lambda}}
	\label{eqn:total-proper-time}
\end{align}
\begin{note}
	The proper time elapsed depends on the path taken between the two events.
\end{note}

Particle worldlines can be characterised by an action principle: variation of \eqref{eqn:total-proper-time} with respect to the trajectories, $\delta x^\mu(\lambda)$, yields the equations of motion $\pdv[2]{x^\mu}{\lambda} = 0$ (holding $\delta x^\mu = 0$ fixed at both ends) \exercise.
In fact, the extremised trajectory minimises the proper time---\emph{freely falling particles take the path of maximum proper time.}

By \eqref{eqn:def-proper-time},
\begin{math}
	c^2\dd\tau^2 = -\dd t^2 + \sum_i(\dd x^i)^2 = \dd t^2\qty(-1 + \uvec v\cdot\uvec v)
\end{math}
where $v^i \equiv \dv{x^i}{t}$ so
\begin{align}
	\dd\tau = \sqrt{1 - \frac{v^2}{c^2}}\dd t \equiv \gamma\dd t
\end{align}
is the proper time of a frame moving at velocity $\uvec v$ relative to a frame with coordinates $(ct, x^i)$.