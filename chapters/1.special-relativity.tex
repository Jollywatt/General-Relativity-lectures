\chapter[Review of Special Relativity]{Review of Special Relativity}

\begin{note}[Background]
See \cite[ch~1]{schutz2009first} and \cite[ch 5, 12, 13]{doughty2018lagrangian}.
\end{note}

\noindent
Assumptions of Special Relativity:
\begin{enumerate}
	\item The world is described by a 4-dimensional continuum\footnote{Mathematically, a \emph{Lorentzian manifold}...}, \textbf{spacetime}, or \textbf{Minkowski space} $\Mink^4$, which is the set of all \textbf{events} $x^μ$,
	\begin{align}
		x^μ \equiv \ts x = \qty(x^0, x^i) \equiv \qty(x^0, \uvec x) = \underset{\text{time, space}}{\qty(ct, \uvec x)}
	.\end{align}
	\begin{note}[Notation]
	Greek indices, $μ, ν$ run over \textit{spacetime} index values; $\qty{0,1,2,3}$. \\
	Latin indices (mid-alphabet), $i,j,k$ run over \textit{spatial} index values; $\qty{1,2,3}$.
	$\ts x$ is a 4-vector (twiddle); $\uvec x$ is a 3-vector (under bar).
	\end{note}
	
	\item There exist \textbf{inertial frames}; namely, frames in which the measured values of time $t$ and position $x^i$ of events result in \textit{linear} equations of motion for \textit{free} particles.
\end{enumerate}


\noindent
Postulates of Special Relativity:
\begin{enumerate}
	\item \textsc{Principle of Relativity:}
	The laws of physics are invariant under transformations $x^ν \to x^{\bar μ}(x^ν)$ from one inertial frame to another (and such transformations form a group).
	
	\item \textsc{Constancy of the Speed of Light:} There exists an invariant upper bound on all velocities
	\begin{align}
		\qty|\dv{\uvec x}{t}| \le \qty|\dv{\uvec x}{t}|_\text{max} = c
		\qqtext{(speed of light)}
	\end{align}
	and this value is the same in all inertial frames.
\end{enumerate}
	
For photons,
\begin{align}
	c = \underset{\text{frame $K$}}{\qty|\dv{\uvec x}{t}|} = \underset{\text{frame $K'$}}{\qty|\dv{\uvec x'}{t}|}
	\label{eqn:photon-speed}
.\end{align}
Rewriting \eqref{eqn:photon-speed} we see that, for photons,
\begin{align}
	\underset{\text{frame $K$}}{(\dd x)^2 + (\dd y)^2 + (\dd z)^2 - c^2(\dd t)^2}
	= \underset{\text{frame $K'$}}{(\dd x')^2 + (\dd y')^2 + (\dd z')^2 - c^2(\dd t')^2} = 0
.\end{align}
This suggests the definition of the \textbf{spacetime interval} between any neighbouring events $x^μ$ and $x^μ + \dd x^μ$ as
\begin{align}
	\dd s^2 &\equiv -c^2\dd t^2 + \dd x^2 + \dd y^2 + \dd z^2
\\  &= η_{μν}\dd x^μ \dd x^ν
	\eqnote{...pseudo-Riemannian structure}
\intertext{where $x^μ = (x^0, x^i) = (ct, x, y, z)$ and where}
	η_{μν} &= \smqty(\dmat{-1,+1,+1,+1})
	\equiv \operatorname{diag}\qty(-1, +1, +1, +1)
	\equiv -1 \oplus \mathds{1}_3
.\end{align}
For photons, $\dd s^2 = 0$.
Using the two postulates, one may show that the interval $\dd s^2$ is invariant with respect to coordinates based in any inertial frame \cite[\S1.6]{schutz2009first}.
\begin{align}
	\dd s^2 =η_{μν}\dd x^μ \dd x^ν = η_{\bar μ\bar ν}\dd x^{\bar μ} \dd x^{\bar ν}
	\label{eqn:def-spacetime-interval}
.\end{align}

\begin{note}
\begin{itemize}%[label=\arabic*)]
	\item The symbol $\dd s^2$ for the interval is purely notational convention, since we may have $\dd s^2 < 0$ is some cases.

	\item The first postulate alone implies either S.R.\ or its $c \to \infty$ limit (a.k.a.\ Galilean relativity).

	\item Formally, $\dd x^μ\dd x^ν$ is shorthand for the \textit{tensor product} $\dd x^μ \otimes \dd x^ν$, and you will see that $\dd s^2 = \tsη$ is really the \emph{metric tensor}...
\end{itemize}
\end{note}


\section{Lorentz Transformations}

In an inertial frame $K$, the equations of motion of a free particle are linear;
\begin{align}
	\dv[2]{x^μ}{λ} &= 0
&   &\iff
&   x^μ &= x_0^μ + u^μλ
,\end{align}
where $x_0^μ, u^μ$ are constant and $λ$ is a parameter.
Similarly, in any other inertial frame $\bar K$,
\begin{align}
	\dv[2]{x^{\bar μ}}{λ} &= 0
&   &\iff
&   x^{\bar μ} &= x_0^{\bar μ} + u^{\bar μ}λ
.\end{align}
Now,
\begin{align}
	\dv{x^{\bar μ}}{λ} &= \pdv{x^{\bar μ}}{x^ν}\dv{x^ν}{λ} = \pdv{x^{\bar μ}}{x^ν}u^ν
	\eqnote{...chain rule}
\\\implies
	0 &= \dv[2]{x^{\bar μ}}{λ} = \pdv{x^α}\qty(\pdv{x^{\bar μ}}{x^ν}u^ν)u^α
	= \pdv[2]{x^{\bar μ}}{x^α}{x^ν}u^ν u^α
	\eqnote{$\because \; u^ν$ constant}
\\\implies 0 &= \pdv[2]{x^{\bar μ}}{x^α}{x^ν}
	\eqnote{...since true $\forall u^α$}
.\end{align}
So the required transformation between inertial frames is \textit{linear};
\begin{eqbox}
	x^{\bar μ} = L^{\bar μ}{}_ν x^ν + a^{\bar μ}
	\label{eqn:Poincare-coord-transform}
,\end{eqbox}
where $a^{\bar μ}$ and $L^{\bar μ}{}_ν \equiv \pdv{x^{\bar μ}}{x^nu}$ are constants.
Differentiate \eqref{eqn:Poincare-coord-transform} and substitute into \eqref{eqn:def-spacetime-interval} to give
\begin{math}
	η_{μν}\dd x^μ\dd x^ν
	= η_{\bar μ\bar ν}L^{\bar μ}{}_μ L^{\bar ν}{}_ν\dd x^μ\dd x^ν
.\end{math}
Since this is true $\forall \, \dd x^μ$,
\begin{eqbox}
	η_{\bar μ\bar ν}L^{\bar μ}{}_μ L^{\bar ν}{}_ν
	= η_{μν}
	\label{eqn:Lorentz-matrix-condition}
.\end{eqbox}
In matrix form, \eqref{eqn:Poincare-coord-transform} and \eqref{eqn:Lorentz-matrix-condition} are
\begin{eqbox}[gather]
	\ts{\bar x} = L\ts{x} + \ts{a} \tag{\ref{eqn:Poincare-coord-transform}a}
,\\  L\transpose η L = η \tag{\ref{eqn:Lorentz-matrix-condition}a}
.\end{eqbox}

Transformations $\ts x \mapsto \ts{\bar x}$ defined by \eqref{eqn:Poincare-coord-transform},~\eqref{eqn:Lorentz-matrix-condition} are the \textit{inhomogeneous} Lorentz transformations, or \textbf{Poincaré transformations}, and form the Poincaré group $\mathrm{IO}(1,3)$ (pronounced Inhomogeneous Orthogonal group).



\begin{wrapfigure}[8]{r}{5cm}
	\vspace*{-\baselineskip}
	\begin{tikzpicture}[sketch]

\begin{axis}[
    Frame,
    name=K,
]
\end{axis}

\begin{axis}[
    Barred Frame,
    name=KBAR,
    at={($(K) + (15:3cm)$)},
]
\end{axis}

\draw[->,dashed] (K.center) --node[above] {$\ts a$} (KBAR.center);

\node[anchor=north west] at (.3, -.4) {\small \em (homogeneity of spacetime)};

\end{tikzpicture}
\end{wrapfigure}

``Inhomogeneous'' refers to the inclusion of spacetime translations
\begin{math}
	x^ν \mapsto x^{\bar μ} = δ^{\bar μ}{}_ν x^ν + a^{\bar μ}
,\end{math}
which form a subgroup $\mathrm{T}^4$ of the Poincaré group.
If we set $a^{\bar μ} = 0$ in \eqref{eqn:Poincare-coord-transform}, we are left with \textit{homogeneous} transformations, called simply the \textbf{Lorentz transformations}.


In G.R., our task is to generalise these ideas to general coordinate frames for which $L^{\bar μ}{}_ν$ are not necessarily constant.


\subsection{Examples of Lorentz Transformations}

A transformation belonging to the (homogeneous) Lorentz group $\O(1,3)$ can be represented as a matrix acting on coordinates $x^μ$ when they are \emph{viewed as vectors.}
\begin{align}
	x^μ \cong \ts{x} = \smqty(x^0 \\ x^1 \\ x^2 \\ x^3)
	= \smqty(t \\ x \\ y \\ z)
\end{align}
% Note that spacetime events (i.e., \textit{points} in spacetime) should not generally be thought of as \textit{vectors}---except when the spacetime is flat and the coordinates are expressed relative to a nearby orthogonal frame, as we assume here.

\begin{note}
	Spacetime events (i.e., \textit{points} in spacetime) are not themselves vectors---neither addition nor scalar multiplication of events makes physical sense (i.e., spacetime itself is not a \textit{vector space}).
	However, in S.R.\ we may represent events by their associated displacement vector relative to a chosen orthogonal inertial frame.
\end{note}

\noindent
The Lorentz group $\O(1,3)$ consists of (combinations of) the following:
\begin{itemize}
	\item \textbf{Rotations,} e.g., about the $z$-axis (in the $xy$-plane) by an angle $θ$;
	\begin{align}
		L_R(θ) &= \mqty(
			1 & 0 & 0 & 0 \\
			0 & \cosθ & -\sinθ & 0 \\
			0 & \sinθ & \cosθ & 0 \\
			0 & 0 & 0 & 1 \\
		)
	.&	&\begin{tikzpicture}[baseline=-2ex]

\begin{axis}[
	Frame,
	name=K,
	view={30}{30},
	ylabel style={above right},
	zlabel={$z, \bar z$},
]
\end{axis}

\begin{axis}[
	Barred Frame,
	name=KBAR,
	at=(K),
	view={30}{30},
	% rotate around x=35,
	rotate around z=-30,
	xlabel style={below},
	ylabel style={right},
	zlabel={},
]
\end{axis}

\node at (0, -1.2) {\small\em(isotropy of spacetime)};

\end{tikzpicture}
	\end{align}
	
	\item \textbf{Boosts,} e.g., by a velocity $\uvec v$ in the $x$-direction;
	\begin{align}
		L_B(α) = \mqty(
			\coshα & -\sinhα & 0 & 0 \\
			-\sinhα & \coshα & 0 & 0 \\
			0 & 0 & 1 & 0 \\
			0 & 0 & 0 & 1 \\
		) = \mqty(
			\cos iα & i\sin iα & 0 & 0 \\
			i\sin iα & \cos iα & 0 & 0 \\
			0 & 0 & 1 & 0 \\
			0 & 0 & 0 & 1 \\
		)
	,\end{align}
	where \begin{math}
		α = \tanh^{-1}\frac{v}{c}
	\end{math}
	is the \textit{rapidity parameter}.
	In terms of the velocity $β \equiv \frac{v}{c}$, one has $\coshα = \frac1{\sqrt{1 - β^2}} \equiv γ$ and $\sinhα = \frac{β}{\sqrt{1 - β^2}} \equiv βγ$.

	\begin{note}
	Boosts appear similar to rotations, but differ as a consequence of the indefiniteness of the metric: they are \emph{hyperbolic}.
	E.g., consider a vector space with signature $(-+)$; a `rotation' which preserves the inner product has the same form of the $tx$-submatrix in $L_B(α)$.

	Formally, an $x$-boost of rapidity $α$ is equivalent to a rotation by an `imaginary angle' $α' = iα$ through the $τx$-plane, where $τ = it$ is `imaginary time'... though this is not a good picture physically!
	\end{note}

\end{itemize}

\noindent
An important difference between rotations and boosts is that, where $0 \le θ < 2π$ for rotations, we have $-\infty < α < \infty$ for boosts, i.e., rotations form a \textit{compact}\footnote{A \textit{compact} set is one for which any infinite sequence of elements contains a convergent subsequence. E.g., $[0, 2π)$ is compact, but $\mathds{R}$ is not (consider the sequence $\qty{1, 2, 3, ...} \subset \mathds{R}$).} subgroup of the Lorentz (or Poincaré) group, whereas boosts are \textit{non-compact}---and in fact do not form a subgroup (because, in general, the composition of two boosts forms a combination of a rotation and a boost).

Both rotations and boosts depend on continuous parameters ($θ$ or $α$).
However, the Lorentz group $\O(1,3)$ also contains \emph{discrete} transformations...
\begin{itemize}

	\item \textbf{Parity inversion};
	\begin{math}
		P = \operatorname{diag}(1, -\mathds{1}_3) \equiv
		\smqty(1 & & & \\ & -1 & & \\ & & -1 & \\ & & & -1)
	.\end{math}

	{\centering
		\begin{tikzpicture}[sketch]


\begin{axis}[
    Frame,
    name=K,
    view={30}{20},
]
\end{axis}

\begin{axis}[
    Barred Frame,
    name=KBAR,
    at={($(K) + (0:3.5cm)$)},
    view={180+30}{20},
    xlabel style={anchor=east},
    ylabel style={anchor=north east},
    zlabel style={anchor=north},
    z post scale=-1
]
\end{axis}

\draw[->] (K) --node[above] {$P$} (KBAR);

\end{tikzpicture}
	\\}
	Notice that $P$ is \textit{not} equivalent to a rotation; it transforms a right-handed frame into a left and vice versa, since $\det P = -1$.

	\begin{note}
	In even spatial dimensions, the transformation $R = \operatorname{diag}(1, -\mathds{1}_{2n})$ is \textit{not} a parity transformation; it is a rotation by $π$ and has $\det R = 1$.
	In these cases, inversions $x_i \mapsto -x_i$ of a \textit{single} spatial coordinate are each parity transformations.
	\end{note}

	\item \textbf{Time reversal};
	\begin{math}
		T = \operatorname{diag}(-1, \mathds{1}_3)
	.\end{math}
\end{itemize}

The Lorentz matrix condition \eqref{eqn:Lorentz-matrix-condition} implies that $(\det L)^2 = 1 \iff \det L = \pm 1$ for any Lorentz transformation $L \in \O(1,3)$.
Those with $\det L = +1$ and those with $\det L = -1$ form two disconnected pieces of $\O(1,3)$, but only the first piece contains the identity transformation $δ^{\bar μ}{}_ν$.

Inspecting the $\bar μν = \bar00$ component of \eqref{eqn:Lorentz-matrix-condition} gives
\begin{align}
	\qty(L^{\bar0}{}_0)^2 - \sum_{\bar k = 1}^3\qty(L^{\bar k}{}_0)^2 = 1
	\implies
	\qty(L^{\bar0}{}_0)^2 \ge 1
,\end{align}
which shows that there exists two disconnected classes of Lorentz transformation with $L^{\bar0}{}_0 \ge 1$ and $L^{\bar0}{}_0 \le -1$.
Those with $L^{\bar0}{}_0 \ge 1$ are called \textbf{orthochronous}.

\subsection{The Restricted Lorentz Group}

We define the subgroup of restricted Lorentz transformations by adding two conditions to the Lorentz matrix condition \eqref{eqn:Lorentz-matrix-condition}; that they be
\begin{enumerate*}[label=\arabic*)]
	\item orthochronous and
	\item have determinant unity.
\end{enumerate*}
\begin{align}
	\SO^+(1,3) \equiv \qty{Λ \suchthat Λ\transposeηΛ = η,
	Λ^{\bar0}{}_0 \ge 1, \detΛ = 1}
	\label{eqn:def-restricted-Lorentz-group}
\end{align}
We have removed the discrete transformations involving $ P$ and $ T$, so that $\SO^+(1,3)$ is \emph{continuous} and \emph{connected}, unlike $\O(1,3)$.

\begin{note}[Notation]
	The $\mathrm{S}$ in $\SO^+(1,3)$ refers to the condition $\detΛ = 1$, and the $^+$ refers to orthochronality.
	Sometimes $\SO^+(1,3)$ is simply written as $\SO(1,3)$.
\end{note}

Groups whose elements may be continuously parametrised are \textbf{Lie groups}.
The translation group $\mathrm{T}^4$ (parametrised continuously by $Δx^μ$) and the rotation group $\SO(3)$ (parametrised continuously by three angles) are examples of \emph{connected} Lie groups.

Reintroducing translations to the restricted Lorentz group gives the restricted Poincaré group $\mathrm{ISO}^+(1,3)$---also a connected Lie group.
Unrestricted groups may be reconstructed by reintroducing the discrete transformations;
\begin{align}
	\O(1,3) = \qty{Λ, Λ P, Λ T, Λ PT \suchthat Λ \in \SO^+(1,3)}
.\end{align}






\section{The Scalar Product in Minkowski Spacetime}

We define a scalar product
\begin{align}
    \dd s^2 = \eta_{\mu\nu} \dd x^\mu \dd x^\nu = \dd x_\mu \dd x^\mu \equiv \ts{\dd x} \cdot \ts{\dd x},
\end{align}
where 
\begin{align}
    \dd x_\mu = \eta_{\mu\nu} \dd x^\nu = \qty(- \dd x^0, \dd x^1, \dd x^2, \dd x^3 ) \label{eqn:dx-lowered}
\end{align}
is the dual component form.

We refer to $\qty{\dd x^\mu}$ as the \textit{contravariant} components, and $\qty{\dd x_\mu}$ as the \textit{covariant} components of the vector $\ts{ \dd x}$.
Thus the metric $\eta_{\mu\nu}$ acts as an \textit{index-lowering operator}\footnote{N.B. $\qty{ \dd x_\mu }$ are \textit{not} differentials of coordinates.}.

The identity operation (a Lorentz transformation of course) is the Kronecker delta $\delta^\mu{}_\nu$ in indexed form.

Thus $\delta^\mu{}_\nu = \qty(\eta^{-1} \eta)^\mu{}_\nu = \qty(\eta^{-1})^{\mu\lambda} \eta_{\lambda\nu}$ by applying the rule that the indices of $\eta$ are downstairs.
As a matrix, $\eta^{-1} = \eta = \operatorname{diag}\qty(-1, \mathds{1}),$
but we write $\eta^{-1} = \qty(\eta^{-1})^{\mu\lambda} = \eta^{\mu\lambda}$ (with indices raised) so that levels of indices are consistent, i.e.
\begin{align}
    \eta^{\mu\lambda} \eta_{\lambda \nu} = \delta^\mu{}_\nu.
    \label{eqn:metric-inverse}
\end{align}
By \eqref{eqn:dx-lowered} and \eqref{eqn:metric-inverse} if follows that
\begin{align}
    \dd x^\mu = \eta^{\mu\lambda} \dd x_\lambda, 
    \label{eqn:eta-raising}
\end{align}
i.e. $\eta^{\mu\lambda}$ is an index-raising operator.

Under a Poincar\'e transformation $x^\mu \to x^{\bar \mu} \qty(x^\mu)$
\begin{align}
    &\dd x^{\bar \mu} = L^{\bar \mu}{}_\mu \dd x^\mu, &\text{where } L^{\bar \mu}{}_\mu = \pdv{x^{\bar \mu}}{x^\mu} \in O(1,3).
    \label{eqn:poincare-transformation}
\end{align}
The inverse transformation taking $x^{\bar \mu}$ back to $x^\mu$ just swaps indices $\bar \mu \leftrightarrow \mu$ in above. 
Thus the natural notation for the Lorentz matrix inverse to $L^{\bar \mu}{}_\mu$ is $\qty(L^{-1})^\mu{}_{\bar \mu} \equiv L^\mu{}_{\bar \mu} = \pdv{x^\mu}{x^{\bar \mu}}$.
By the chain rule:
\begin{align}
    &L^{\bar \mu}{}_\mu L^\mu{}_{\bar \nu} = \delta^{\bar \mu}{}_{\bar \nu}, &L^\mu{}_{\bar \mu} L^{\bar \mu}{}_\nu = \delta^\mu{}_\nu.
\label{eqn:lorentz-chainrule}
\end{align}
From \eqref{eqn:dx-lowered}, \eqref{eqn:poincare-transformation}, and (\ref{eqn:Lorentz-matrix-condition}a) it follows that \exercise under \eqref{eqn:poincare-transformation}
\begin{align}
    \dd x_{\bar \mu} = L^\mu{}_{\bar \mu} \dd x_\mu.
    \label{eqn:dx-lowered-lorentz}
\end{align}


\section{The Causal Structure of Minkowski Spacetime}

Because of the indefiniteness of the Minkowski metric (and hence of the Minkowski scalar product), 4-vectors can have positive, zero or negative norm.
\\
A 4-vector $\ts V$ is called
\begin{math}
	\begin{baligned}
		&\text{timelike}
	\\	&\text{null}
	\\	&\text{spacelike}
	\end{baligned}
\end{math}
if 
\begin{math}
	\ts V \cdot \ts V = V_μ V^μ = η_{μν}V^μ V^ν
	\begin{baligned}
		&< 0
	\\	&= 0
	\\	&> 0
	\end{baligned}
.\end{math}
% Similarly, a spacetime interval $\dd s$ is
% \begin{math}
% 	\begin{cases}
% 		\text{timelike} & \text{if $\dd s^2 < 0$} \\
% 		\text{null} & \text{if $\dd s^2 = 0$} \\
% 		\text{spacelike} & \text{if $\dd s^2 < 0$} \\
% 	\end{cases}
% \end{math}

\begin{wrapfigure}[16]{l}{6cm}
	\begin{tikzpicture}[
	point/.style={
		circle,
		fill,
		inner sep=1pt,
	},
	smol/.style={
		font=\small,
	}
]
	\begin{axis}[
		Frame,
		width=6cm,height=7cm,
		zlabel={$ct$},
		xlabel={$x^i$},
		ylabel={},
		z buffer = sort,
	]
	\addplot3[
		% surf,
		% shader = interp,
		smooth,
		opacity = 0.3,
		mesh,
		samples = 7,
		samples y = 17,
		domain = -1:1,
		domain y = 0:360,
		colormap={wbw}{rgb255=(255,255,255) rgb255=(0,0,0) rgb255=(255,255,255)},
	] ({x*sin(y)}, {x*cos(y)}, {x});

	\node[point] (O) at (axis cs:0,0,0) {};
	\node[point] (F) at (axis cs:0.3,0.3,1) {};
	\node[point] (E) at (axis cs:-0.3,-0.3,0) {};
	\node[point] (P) at (axis cs:-0.3,-0.3,-1) {};
	\end{axis}

	\node[smol,circle, inner sep=1pt, fill=white] at (O) {$P$};
	\node[smol,above right] at (F) {future of $P$};
	\node[smol,left] at (E) {elsewhere of $P$};
	\node[smol,below] at (P) {past of $P$};
\end{tikzpicture}
	\caption*{One spatial dimension suppressed; lightcone in $(3+1)$-d spacetime is a continuum of spheres.}
\end{wrapfigure}
By the second postulate, all \emph{causally connected} events relative to an event $P$ lie in its future or past lightcone, because $P$ cannot causally influence events outside the lightcone without transmitting superluminal data.

A particle's \textbf{worldline} is a curve $x^μ = x^μ(λ)$ whose tangent $\ts V$ given by $V^μ = \dv{x^μ}{λ}$ is everywhere timelike; $V^μ V_μ < 0$.
This implies that, in an inertial frame, the particles velocity $\uvec v$ is subluminal $|\uvec v| < c$ \exercise.

Consider a freely moving particle whose (timelike) worldline unit tangent vector is $\ts V$.
There always exists a Lorentz transformation which sends a timelike unit vector to $V^μ \mapsto V^{\bar μ} = (1, 0, 0, 0)$ \exercise.
The \textbf{proper time interval} $\ddτ$ between two neighbouring events along the particle's worldline is defined as the interval of time measured in the particle's instantaneous rest frame.
\begin{align}
	c^2\ddτ^2 = -η_{μν}\dd x^μ\dd x^ν
	\label{eqn:def-proper-time}
\end{align}

\begin{note}
	Since $\dd x$ is timelike, $η_{μν}\dd x^μ\dd x^ν < 0$, so \eqref{eqn:def-proper-time} contains a minus sign.
	Proper time is not defined for spacelike intervals, since this would yield an imaginary time.
	Instead, proper distance is defined by $\dd\ell^2 = η_{μν}\dd x^μ\dd x^ν$.
\end{note}
Proper time is the time \emph{experienced} by the particle, i.e., the time that would be measured by a clock moving on the same worldline.

Suppose we have a worldline $x^μ(λ)$.
Integrate $\ddτ$ via \eqref{eqn:def-proper-time} along the path to get the total proper time elapsed between events $x^μ(λ_i)$ and $x^μ(λ_f)$.
\begin{align}
	\Deltaτ = \frac1c \int_{λ_i}^{λ_f} \ddλ \, \sqrt{-η_{μν}\dv{x^μ}{λ}\dv{x^ν}{λ}}
	\label{eqn:total-proper-time}
\end{align}
\begin{note}
	The proper time elapsed depends on the path taken between the two events.
	(See the Twin Paradox.)
\end{note}

Particle worldlines can be characterised by an action principle: variation of \eqref{eqn:total-proper-time} with respect to the trajectories, $δx^μ(λ)$, yields the equations of motion $\pdv[2]{x^μ}{λ} = 0$ (holding $δx^μ = 0$ fixed at both ends) \exercise.
In fact, the extremised trajectory minimises the proper time---\emph{freely falling particles take the path of maximum proper time.}

By \eqref{eqn:def-proper-time},
\begin{math}
	c^2\ddτ^2 = -\dd t^2 + \sum_i(\dd x^i)^2 = \dd t^2\qty(-1 + \uvec v\cdot\uvec v)
\end{math}
where $v^i \equiv \dv{x^i}{t}$ so
\begin{align}
	\ddτ = \sqrt{1 - \frac{v^2}{c^2}}\dd t \equiv γ\dd t
	\label{eqn:gamma-factor}
\end{align}
is the proper time of a frame moving at velocity $\uvec v$ relative to a frame with coordinates $(ct, x^i)$.


\subsection{4-velocity}

Since proper time $\ddτ$ is an \emph{invariant}, we are motivated to define the 4-vector
\begin{align}
	u^μ \equiv \dv{x^μ}{τ}
	= \dv{t}{τ}\qty(c, \dv{\uvec x}{t})
	= γ(c, \uvec v)
,\end{align}
where $γ$ is defined as in \eqref{eqn:gamma-factor}.
In the particle's instantaneous rest frame, $\|\uvec v\|^2 = u^iu_i = 0$, so that $u^μ u_μ = u^0u_0 = -c^2$.
However, since this is an invariant (it is a scalar, which does not transform under Poincaré transformations), we have
\begin{eqbox}
	\|\ts u\|^2 \equiv \ts u \cdot \ts u \equiv u^μ u_μ = -c^2
\end{eqbox}
in \emph{all frames}.


\subsection{4-momentum}

Relativistic 4-momentum is defined by
\begin{align}
	\ts p \equiv m\ts u &= \qty(\frac{E}{c}, \uvec p)
,&	E &= γmc^2
,&	\uvec p &= γm\uvec v
	\label{eqn:4-momentum}
.\end{align}
This form of 4-momentum may be deduced from an action using the Hamiltonian approach.
We want to extremise \eqref{eqn:def-spacetime-interval}, but first multiply by $mc^2$ to get the units right.
The action is
\begin{align}
	S = -mc^2\int_{\ts x_i}^{\ts x_f} \ddτ = \int_{t_i}^{t_f} \dd t L
\end{align}
where the Lagrangian is
\begin{align}
	L = L(\uvec x, \dot{\uvec x}; t)
	= -mc^2\sqrt{1 - \frac{v^2}{c^2}}
	= -mc^2\sqrt{1 - \frac{\uvec x \cdot \uvec x}{c^2}}
.\end{align}
Then, the Hamiltonian (total energy) and conjugate momentum are
\begin{align}
	E &= \uvec p\cdot\uvec x - L = γ mv^2 + \frac{mc^2}{γ} = γ mc^2
,\\	\uvec p &= \pdv{L}{\dot{\uvec x}}
	= mc^2\frac12\frac{2\dot{\uvec x}}{\sqrt{1 - v^2/c^2}}
	= γ m\uvec v
,\end{align}
matching the definition \eqref{eqn:4-momentum}.

In all rest frames,
\begin{eqbox}
	\|\ts p\|^2 = \ts p\cdot\ts p = p^μ p_μ = -m^2c^2
\end{eqbox}
as can be shown by evaluating in the particle's instantaneous rest frame and using scalar invariance.

For photons with $m = 0$, the 4-momentum $\ts p$ is a null vector.
Photons have no instantaneous rest frame---but we can always find a frame in which $\ts p = \frac{E}{c}(1, 0, 0, 1)$.


\subsection{Angular momentum}


The covariant orbital angular momentum of a particle about an event $\ts x_0 = \qty{x_0^μ}$ is defined as
\begin{align}
	j^{μν} = (x^μ - x_0^μ)p^ν - (x^ν - x_0^ν)p^μ
,\end{align}
where $x^μ$ and $p^μ$ are the particle's position and momentum.
This is an antisymmetric tensor,
\begin{math}
	j^{μν} = -j^{νμ} = j^{[μν]}
,\end{math}
with 6 independent components: $j^{0k}, j^{k\ell}, k < \ell$.
\begin{note}[Example]
	About the origin $x_0^μ = 0$, we have
	\begin{math}
		j^{ij} = x^ip^j - x^jp^i
	\end{math}
	which is equivalent to $j^{ij} = \leviciv^{ijk}\ell_k$ where $\uvec l = \uvec x \times \uvec p$.
	% $\uvec l$ is a vector, whereas $j^
	\alert{which is a vector, pseudovector, bivector?}
	$j^{0k}$ represents the motion of the centre of mass:
	\begin{math}
		j^{0k} = x^0p^k - x^kp^0 = γmc(v^kt - x^k)
	.\end{math}
\end{note}







\section{Gauß's Theorem in Minkowski Spacetime}

In three dimensional Euclidean space, Gauß's theorem is
\begin{align}
	\int_R \dd^3x \, \nabla \cdot \uvec V = \oint_{\partial R} \dd \uvec S \cdot \uvec V
,\end{align}
where $\uvec V$ is a 3-vector and $R \subset \mathds{R}^3$ is a region of Euclidean space with boundary $\partial R$.
The surface element $\dd\uvec S$ is normal to the boundary $\partial R$.
In index form,
\begin{align}
	\int_R\dd^3x \, \partial_iV^{i...} = \oint_{\partial R} \dd S_i V^{i...}
,\end{align}
where dots denote possible free indices, which are not involved.
The proof of Gauß's theorem does not require 3 dimensions or a $+++$ metric signature.
In Minkowski spacetime $\Mink^4$, Gauß's theorem reads
\begin{align}
	\int_R\dd^4x \, \partial_μ V^{μ...} = \oint_{\partial R} \dd Σ_μV^{μ...}
,\end{align}
where $\partial R$ is now the boundary of a simply-connection region $R \subset \Mink^4$ with normal surface element $\dd\tsΣ$.


In 4-dimensional spacetime, we often choose the boundary $\partial R$ so that it has three parts: $Σ^+$, spacelike surface toward future; $Σ^-$, spacelike surface toward past; $\partial R^\infty$, timelike surface at spatial infinity, joining $Σ^+$ and $Σ^-$.
\begin{note}
	The outward normal to $Σ^-$ is \emph{past-pointing}.
\end{note}
This decomposition is convenient if the fields involved vanish at spatial infinity, in which case the contribution from $\partial R^\infty$ vanishes.

\section{Conserved Currents}

A 4-vector field $\ts J = J^μ\partial_μ$ which is divergence free
\begin{align}
	\operatorname{div}\ts J \equiv \pdv{J^μ}{x^μ} \equiv \partial_μ J^μ \equiv J^μ{}_{,μ} = 0
	\label{eqn:divergenceless}
\end{align}
is called a \textbf{4-current}. Equation \eqref{eqn:divergenceless} is a conservation law.
By Gauß's theorem, supposing that contributions at spatial infinity vanish,
\begin{align}
	0 = \int_R \dd^4 x \, \partial_μ J^μ = \int_{Σ^-} \dd Σ_μ J^μ + \int_{Σ^+} \dd Σ_μ J^μ 
.\end{align}
If we define $Σ_f \coloneqq Σ^+$ and $Σ_p$ to be $Σ^-$ but future-pointing instead of past-pointing, i.e., $\int_{Σ_p}\dd Σ_μ J^μ = -\int_{Σ_-}\dd Σ_μ J^μ$, then the equation above becomes
\begin{align}
	\int_{Σ_p}\dd Σ_μ J^μ = \int_{Σ_f}\dd Σ_μ J^μ = \int_{t\ \const}\dd^3 x\,J^0
\end{align}